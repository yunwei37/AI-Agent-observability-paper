\documentclass[sigconf,10pt]{acmart}
\pdfoutput=1

% \renewcommand\footnotetextcopyrightpermission[1]{}
% \settopmatter{printacmref=false, printccs=false, printfolios=false}

\usepackage{listings}     % For ASCII-art / code blocks
\usepackage{booktabs}     % Nicer tables
\usepackage{array}        % Column types
\usepackage{tabularx}     % Automatic column width
\usepackage{enumitem}     % Compact lists


\usepackage{comment}

\usepackage[utf8]{inputenc}
\usepackage[T1]{fontenc}
\usepackage{textcomp}
\usepackage[english]{babel} 

\DeclareUnicodeCharacter{00A0}{~}      % Non-breaking space
\DeclareUnicodeCharacter{00A1}{\textexclamdown}
\DeclareUnicodeCharacter{00A2}{\textcent}
\DeclareUnicodeCharacter{00A3}{\textsterling}
\DeclareUnicodeCharacter{00A5}{\textyen}
\DeclareUnicodeCharacter{00A7}{\S}
\DeclareUnicodeCharacter{00A8}{\"{}}
\DeclareUnicodeCharacter{00A9}{\textcopyright}
\DeclareUnicodeCharacter{00AA}{\textordfeminine}
\DeclareUnicodeCharacter{00AB}{\guillemotleft}
\DeclareUnicodeCharacter{00AC}{\textlnot}
\DeclareUnicodeCharacter{00AD}{}       % Soft hyphen
\DeclareUnicodeCharacter{00AE}{\textregistered}
\DeclareUnicodeCharacter{00AF}{\={}}   % Macron

\DeclareUnicodeCharacter{00B0}{\textdegree}
\DeclareUnicodeCharacter{00B1}{\textpm}
\DeclareUnicodeCharacter{00B2}{\textsuperscript{2}}
\DeclareUnicodeCharacter{00B3}{\textsuperscript{3}}
\DeclareUnicodeCharacter{00B4}{\'{}}   % Acute accent
\DeclareUnicodeCharacter{00B5}{\textmu}
\DeclareUnicodeCharacter{00B6}{\P}
\DeclareUnicodeCharacter{00B7}{\textperiodcentered}
\DeclareUnicodeCharacter{00B8}{\c{}}   % Cedilla
\DeclareUnicodeCharacter{00B9}{\textsuperscript{1}}
\DeclareUnicodeCharacter{00BA}{\textordmasculine}
\DeclareUnicodeCharacter{00BB}{\guillemotright}
\DeclareUnicodeCharacter{00BC}{\textonequarter}
\DeclareUnicodeCharacter{00BD}{\textonehalf}
\DeclareUnicodeCharacter{00BE}{\textthreequarters}
\DeclareUnicodeCharacter{00BF}{\textquestiondown}

% Accented letters (common in names)
\DeclareUnicodeCharacter{00C0}{\`A}
\DeclareUnicodeCharacter{00C1}{\'A}
\DeclareUnicodeCharacter{00C2}{\^A}
\DeclareUnicodeCharacter{00C3}{\~A}
\DeclareUnicodeCharacter{00C4}{\"A}
\DeclareUnicodeCharacter{00C5}{\AA}
\DeclareUnicodeCharacter{00C6}{\AE}
\DeclareUnicodeCharacter{00C7}{\c{C}}
\DeclareUnicodeCharacter{00C8}{\`E}
\DeclareUnicodeCharacter{00C9}{\'E}
\DeclareUnicodeCharacter{00CA}{\^E}
\DeclareUnicodeCharacter{00CB}{\"E}
\DeclareUnicodeCharacter{00CC}{\`I}
\DeclareUnicodeCharacter{00CD}{\'I}
\DeclareUnicodeCharacter{00CE}{\^I}
\DeclareUnicodeCharacter{00CF}{\"I}
\DeclareUnicodeCharacter{00D1}{\~N}
\DeclareUnicodeCharacter{00D2}{\`O}
\DeclareUnicodeCharacter{00D3}{\'O}
\DeclareUnicodeCharacter{00D4}{\^O}
\DeclareUnicodeCharacter{00D5}{\~O}
\DeclareUnicodeCharacter{00D6}{\"O}
\DeclareUnicodeCharacter{00D8}{\O}
\DeclareUnicodeCharacter{00D9}{\`U}
\DeclareUnicodeCharacter{00DA}{\'U}
\DeclareUnicodeCharacter{00DB}{\^U}
\DeclareUnicodeCharacter{00DC}{\"U}
\DeclareUnicodeCharacter{00DD}{\'Y}
\DeclareUnicodeCharacter{00DF}{\ss}

\DeclareUnicodeCharacter{00E0}{\`a}
\DeclareUnicodeCharacter{00E1}{\'a}
\DeclareUnicodeCharacter{00E2}{\^a}
\DeclareUnicodeCharacter{00E3}{\~a}
\DeclareUnicodeCharacter{00E4}{\"a}
\DeclareUnicodeCharacter{00E5}{\aa}
\DeclareUnicodeCharacter{00E6}{\ae}
\DeclareUnicodeCharacter{00E7}{\c{c}}
\DeclareUnicodeCharacter{00E8}{\`e}
\DeclareUnicodeCharacter{00E9}{\'e}
\DeclareUnicodeCharacter{00EA}{\^e}
\DeclareUnicodeCharacter{00EB}{\"e}
\DeclareUnicodeCharacter{00EC}{\`i}
\DeclareUnicodeCharacter{00ED}{\'i}
\DeclareUnicodeCharacter{00EE}{\^i}
\DeclareUnicodeCharacter{00EF}{\"i}
\DeclareUnicodeCharacter{00F1}{\~n}
\DeclareUnicodeCharacter{00F2}{\`o}
\DeclareUnicodeCharacter{00F3}{\'o}
\DeclareUnicodeCharacter{00F4}{\^o}
\DeclareUnicodeCharacter{00F5}{\~o}
\DeclareUnicodeCharacter{00F6}{\"o}
\DeclareUnicodeCharacter{00F8}{\o}
\DeclareUnicodeCharacter{00F9}{\`u}
\DeclareUnicodeCharacter{00FA}{\'u}
\DeclareUnicodeCharacter{00FB}{\^u}
\DeclareUnicodeCharacter{00FC}{\"u}
\DeclareUnicodeCharacter{00FD}{\'y}
\DeclareUnicodeCharacter{00FF}{\"y}

% Common punctuation
\DeclareUnicodeCharacter{2013}{--}       % en dash
\DeclareUnicodeCharacter{2014}{---}      % em dash
\DeclareUnicodeCharacter{2018}{`}        % left single quote
\DeclareUnicodeCharacter{2019}{'}        % right single quote / apostrophe
\DeclareUnicodeCharacter{201C}{``}       % left double quote
\DeclareUnicodeCharacter{201D}{''}       % right double quote
\DeclareUnicodeCharacter{2022}{\textbullet}
\DeclareUnicodeCharacter{2026}{\ldots}   % ellipsis

% Currency and symbols
\DeclareUnicodeCharacter{20AC}{\euro}    % euro
\DeclareUnicodeCharacter{2122}{\texttrademark}




\begin{document}

\acmYear{2025}\copyrightyear{2025}
\setcopyright{cc}
\setcctype[4.0]{by}
\acmConference[PACMI '25]{Practical Adoption Challenges of ML for Systems}{October 13--16, 2025}{Seoul, Republic of Korea}
\acmBooktitle{Practical Adoption Challenges of ML for Systems (PACMI '25), October 13--16, 2025, Seoul, Republic of Korea}
\acmDOI{10.1145/3766882.3767169}
\acmISBN{979-8-4007-2205-9/25/10}

\title{AgentSight: System-Level Observability for AI Agents Using eBPF}

    
\author{Yusheng Zheng}
% \authornote{Corresponding author.}
\affiliation{%
  \institution{UC Santa Cruz}
  \city{Santa Cruz}
  \state{CA}
  \country{USA}}
\email{yzhen165@ucsc.edu}

\author{Yanpeng Hu}
\affiliation{%
  \institution{ShanghaiTech University}
  \city{Shanghai}
  \country{China}}
\email{huyp@shanghaitech.edu.cn}

\author{Tong Yu}
\affiliation{%
  \institution{eunomia-bpf Community}
  \country{China}
}
\email{yt.xyxx@gmail.com}

\author{Andi Quinn}
\affiliation{%
  \institution{UC Santa Cruz}
  \city{Santa Cruz}
  \state{CA}
  \country{USA}}
\email{aquinn1@ucsc.edu}


\sloppy
\begin{abstract}
    Modern software infrastructure increasingly relies on LLM agents for development and maintenance, such as Claude Code and Gemini-cli. However, these AI agents differ fundamentally from traditional deterministic software, posing a significant challenge to conventional monitoring and debugging. This creates a critical semantic gap: existing tools observe either an agent's high-level intent (via LLM prompts) or its low-level actions (e.g., system calls), but cannot correlate these two views. This blindness makes it difficult to distinguish between benign operations, malicious attacks, and costly failures. We introduce AgentSight, an AgentOps observability framework that bridges this semantic gap using a hybrid approach. Our approach, \emph{boundary tracing}, monitors agents from outside their application code at stable system interfaces using eBPF. AgentSight intercepts TLS-encrypted LLM traffic to extract semantic intent, monitors kernel events to observe system-wide effects, and causally correlates these two streams across process boundaries using a real-time engine and secondary LLM analysis. This instrumentation-free technique is framework-agnostic, resilient to rapid API changes, and incurs less than 3\% performance overhead. Our evaluation shows AgentSight detects prompt injection attacks, identifies resource-wasting reasoning loops, and reveals hidden coordination bottlenecks in multi-agent systems. AgentSight is released as an open-source project at \url{https://github.com/eunomia-bpf/agentsight}.
\end{abstract}

\keywords{eBPF, LLM, observability, AI Agents}

\begin{CCSXML}
<ccs2012>
   <concept>
       <concept_id>10011007.10011006.10011073</concept_id>
       <concept_desc>Software and its engineering~Software maintenance tools</concept_desc>
       <concept_significance>500</concept_significance>
       </concept>
   <concept>
       <concept_id>10002978.10003006.10011608</concept_id>
       <concept_desc>Security and privacy~Information flow control</concept_desc>
       <concept_significance>100</concept_significance>
       </concept>
   <concept>
       <concept_id>10003033.10003099.10003105</concept_id>
       <concept_desc>Networks~Network monitoring</concept_desc>
       <concept_significance>300</concept_significance>
       </concept>
 </ccs2012>
\end{CCSXML}

\ccsdesc[500]{Software and its engineering~Software maintenance tools}
\ccsdesc[100]{Security and privacy~Information flow control}
\ccsdesc[300]{Networks~Network monitoring}


% LLM agents such as Claude Code violate the fundamental assumptions of software monitoring. Their ability to dynamically generate code and spawn arbitrary subprocesses creates a critical semantic gap: existing tools can see either an agent's high-level intent (via LLM prompts) or its low-level actions (via system calls), but never both in a correlated view. This blindness makes it impossible to distinguish between benign operations and malicious attacks or costly failures. We introduce AgentSight, an observability framework designed to bridge this semantic gap. Our approach, \emph{boundary tracing}, monitors agents from outside their application code at stable system interfaces using eBPF. AgentSight intercepts TLS-encrypted LLM traffic to understand semantic intent and monitors kernel events to observe system-wide effects, then causally correlates these two streams across process boundaries. This instrumentation-free technique is framework-agnostic, resilient to rapid API changes, and incurs less than 3\% performance overhead. Our evaluation shows AgentSight can help detect sophisticated prompt injection attacks, identify resource-wasting reasoning loops, and reveal hidden coordination bottlenecks in multi-agent systems. We release AgentSight as open source to enable the safe and observable deployment of autonomous AI in production.


\settopmatter{printfolios=true}


\maketitle
\pagestyle{plain}


\section{Introduction}

The role of machine learning in systems is undergoing a fundamental shift from optimizing well-defined tasks, such as database query planning, to a new paradigm of \emph{agentic computing}. From a systems perspective, an AI agent couples a Large Language Model's (LLM) reasoning with direct access to system tools, granting it agency to perform operations like spawning processes, modifying the filesystem, and executing commands. This technology is being rapidly integrated into production environments, powering autonomous developer tools like Claude Code\cite{claudecode}, Cursor Agent\cite{cursor} and Gemini-CLI\cite{geminicli}, which can independently handle complex software engineering and system maintenance tasks. In essence, we are deploying non-deterministic ML systems, creating an unprecedented class of challenges for system reliability, security, and verification.

This paradigm shift creates a critical semantic gap: the chasm between an agent's high-level \emph{intent} and its low-level \emph{system actions}. Unlike traditional programs with predictable execution paths, agents use LLMs and autonomous tools to dynamically generate code and spawn arbitrary subprocesses. This makes it hard for existing observability tools to distinguish benign operations from catastrophic failures. Consider an agent tasked with code refactoring that, due to a malicious prompt it reads from external url in the search result when search for API documents, instead injects a backdoor (indirect prompt injection)\cite{indirect-prompt-inject}. An application-level monitor might see a successful "execute script" tool call, while a system monitor sees a \texttt{bash} process writing to a file. Neither can bridge the gap to understand that a benign intention has been twisted into a malicious action, rendering them effectively blind.

Current approaches are trapped on one side of this semantic gap. \emph{Application-level instrumentation}, found in frameworks like LangChain~\cite{langchain} and AutoGen~\cite{autogen}, captures an agent's reasoning and tool selection. While these tools see the \emph{intent}, they are brittle, require constant API updates, and are easily bypassed: a single shell command escapes their view, breaking the chain of visibility under a flawed trust model. Conversely, \emph{generic system-level monitoring} sees the \emph{actions}, tracking every system call and file access. However, it lacks all semantic context. To such a tool, an agent writing a data analysis script is indistinguishable from a compromised agent writing a malicious payload. Without understanding the preceding LLM instructions, the \emph{why} behind the \emph{what}, its stream of low-level events is meaningless noise.

We propose {boundary tracing} as a novel observability method designed specifically to bridge this semantic gap. Our key insight is that while agent internals and frameworks are volatile, the interfaces through which they interact with the world (the kernel for system operations and the network for communication) are stable and unavoidable. By monitoring from outside the application at these boundaries, we can capture an agent's high-level intent and its low-level system effects. We present \textbf{AgentSight}, a system that realizes boundary tracing using eBPF to intercept TLS-encrypted LLM traffic for intent and monitor kernel events for effects. Its core is a novel, two-stage correlation process: a real-time engine links an LLM response to the system bahavior it triggers, and a secondary "observer" LLM performs a deep semantic analysis on the resulting trace to infer risks and explain \emph{why} a sequence of events is suspicious. This instrumentation-free, framework-agnostic technique incurs less than 3\% overhead and effectively detects prompt injection attacks, resource-wasting reasoning loops, and multi-agent system bottlenecks.

In summary, our contributions are:

\begin{enumerate}
\item We introduce boundary tracing as a principled approach to AI agent observability that bridges the semantic gap by monitoring at stable system interfaces.
\item We present a novel engine that combines real-time, eBPF-based signal matching with LLM-based semantic analysis to provide deep, contextual understanding of agent behavior.
\item We demonstrate AgentSight's effectiveness in detecting prompt injection attacks, reasoning loops, and multi-agent coordination failures with sub-3\% overhead.
\end{enumerate}

% This semantic gap makes it impossible for existing observability tools to distinguish between benign operations and catastrophic failures. Current approaches fall into two categories, each blind to one side of the gap. {Application-level instrumentation}, found in frameworks like LangChain~\cite{langchain} and AutoGen~\cite{autogen}, uses hooks and logs to capture the agent's reasoning and tool selections. While these tools see the \emph{intent}, they are fundamentally limited. They are brittle, requiring constant updates to keep pace with framework APIs that see dozens of breaking changes monthly. More critically, they are easily bypassed; a single shell command spawned by an agent escapes their view, breaking the chain of visibility. They operate on a flawed trust model, assuming a cooperative agent that will not be compromised or exhibit emergent, unlogged behaviors.

% On the other side, {generic system-level monitoring} tools see the \emph{actions}. They can track every system call, file access, and network connection. However, they lack all semantic context. To them, an agent writing a data analysis script to \texttt{/tmp} is indistinguishable from a compromised agent writing a malicious payload to the same location. Without understanding the preceding LLM instructions, the \emph{why} behind the \emph{what}, their stream of low-level events is meaningless noise, leading to an overwhelming volume of false positives and inactionable alerts.

\section{Background and Related Work}

This section outlines LLM agent architecture, reviews existing observability work to highlight the semantic gap, and introduces eBPF as our foundational technology.

% \subsection{LLM Agent Architecture}
% AI agents represent a new class of software systems that combine language models with environmental interactions. These systems typically consist of three core components: (1) an LLM backend that provides reasoning capabilities, (2) a tool execution framework that enables system interactions, and (3) a control loop that orchestrates prompts, tool calls, and state management. Popular frameworks such as LangChain~\cite{langchain}, AutoGen~\cite{autogen}, cursor agent\cite{cursor}, genimi-cli\cite{geminicli} and Claude Code\cite{claudecode} implement variations of this architecture and are widely used in production. The key characteristic distinguishing AI agents from traditional software is their ability to dynamically construct execution plans based on natural language objectives (e.g., an agent tasked with "analyze this dataset" might autonomously decide to install packages, write analysis scripts, execute them, and interpret results), all without predetermined logic paths. This flexibility comes from the LLM's ability to generate arbitrary code and command sequences.

\subsection{LLM Agent Architecture}
The agentic systems described in the introduction are typically implemented using a common architecture. These systems consist of three core components: (1) an LLM backend for reasoning, (2) a tool execution framework for system interactions, and (3) a control loop that orchestrates prompts, tool calls, and state management. Popular frameworks such as LangChain~\cite{langchain}, AutoGen~\cite{autogen}, cursor agent\cite{cursor}, genimi-cli\cite{geminicli} and Claude Code\cite{claudecode} all implement variations of this model. This architecture is what enables agents to dynamically construct and execute complex plans (e.g., autonomously writing and running a script to analyze a dataset) based on high-level natural language objectives.

\subsection{Observability for LLM Agent}

Existing approaches are siloed on one side of the semantic gap. Intent-side observability, supported by industry tools like Langfuse, LangSmith, and Datadog~\cite{Maierhofer2025Langfuse, langfuse, langsmith, Datadog2023Agents, helicone} and is unifying by standards from the OpenTelemetry GenAI working group~\cite{Liu2025OTel,Bandurchin2025Uptrace} and acadamics conceptual taxonomies~\cite{Dong2024AgentOps, Moshkovich2025Pipeline} under the AgentOps concept, excels at tracing application-level events but is fundamentally blind to out-of-process system \emph{actions}. Conversely, action-side observability with tools like Falco and Tracee~\cite{falco, tracee} offers comprehensive visibility into system calls but lacks the semantic context to understand an agent's \emph{intent}, failing to distinguish a benign task from a malicious one. A parallel line of research into reasoning-level and interpretability aims to make the agent's internal thought processes more transparent by reconstructing cognitive traces~\cite{Rombaut2025Watson} or enabling explanatory dialogues~\cite{Kim2025AgenticInterp}, but these work mainly focus on the llm itself, does not bridge the gap between the agent's internal reasoning and its external, low-level effects on the system.

\subsection{extended Berkeley Packet Filter (eBPF)}

To bridge the semantic gap, our approach requires a technology that can safely and efficiently observe both network communication and kernel activity. eBPF (extended Berkeley Packet Filter) is a fundamental advancement in kernel programmability that provides precisely this capability~\cite{brendangregg}. Originally designed for packet filtering, eBPF has evolved into a general-purpose, in-kernel virtual machine that powers modern observability and security tools~\cite{ebpfio,cilium}. For AI agent observability, eBPF is uniquely suited because it allows observation at the exact boundaries where agents interact with the world—enabling both TLS interception for semantic \emph{intent} and syscall monitoring for system \emph{actions} with minimal overhead. Critically, its kernel-enforced safety guarantees, including verified termination and memory safety, make it ideal for production environments and provide a stable foundation for our solution~\cite{kerneldoc}.


% \subsection{Observability for LLM Agent}


% Observability has historically been critical for building trustworthy and maintainable software systems. Early research, such as Shortliffe et al. pioneering work on the MYCIN expert system in 1975, highlighted the importance of transparent reasoning, establishing foundational concepts for interpreting and monitoring agent-based systems~\cite{Shortliffe1975Mycin}. With the advent of modern large language model (LLM)-driven systems, the complexity and autonomy of AI agents have dramatically increased, amplifying the necessity for enhanced observability. Contemporary AI agents frequently exhibit emergent and non-deterministic behaviors, making traditional observability approaches insufficient for ensuring system reliability, safety, and correctness~\cite{Dong2024AgentOps,Noble2025IBM}.

% Recent work in the domain, such as the \emph{AgentOps} taxonomy developed by Dong et al., underscores the necessity of capturing detailed artifacts such as prompts, tool invocations, and memory interactions as first-class telemetry data for comprehensive observability~\cite{Dong2024AgentOps}. Moshkovich and Zeltyn expand upon this perspective by proposing a structured six-stage automation pipeline that leverages observational data to enable automated analysis and mitigation strategies, thus addressing critical operational challenges when observability data is incomplete or unavailable in production scenarios~\cite{Moshkovich2025Pipeline}.

% Furthermore, frameworks such as \emph{Watson} proposed by Rombaut et al. delve deeper into cognitive-level observability by reconstructing an LLM's implicit reasoning traces, thereby making the agent's internal cognitive processes explicitly visible~\cite{Rombaut2025Watson}. Complementing this, Kim et al. advocate for \emph{agentic interpretability}, wherein LLMs engage in interactive dialogues, actively explaining their reasoning and behaviors to users, effectively positioning the model itself as a cooperative participant in achieving greater transparency and observability~\cite{Kim2025AgenticInterp}.

% From an industry perspective, emerging standards such as those developed by the OpenTelemetry GenAI working group seek to unify observability practices by establishing common semantic conventions for tracing and logging agent actions across diverse frameworks~\cite{Liu2025OTel,Bandurchin2025Uptrace}. Additionally, tools like Langfuse and Datadog provide practical platforms for capturing detailed trace information and visualizing multi-agent workflows, enhancing developers' ability to diagnose and address complex agent interactions effectively~\cite{Maierhofer2025Langfuse,Datadog2023Agents}.

% In sum, the evolving landscape of AI agent observability highlights the critical need for advanced, robust, and integrated observability solutions that accommodate the unique challenges posed by autonomous and dynamically evolving agent behaviors.

% Despite its recognized importance, existing approaches to observability are insufficient due to the special characteristics of AI agents. The autonomy of AI agents creates a fundamental challenge that invalidates traditional monitoring paradigms: the \textbf{semantic gap}. This gap is the chasm between an agent's high-level, semantic \emph{intent} (the "why," captured in LLM interactions) and its low-level system \emph{actions} (the "what," observed as system calls). This chasm is created by two core agent behaviors: \textbf{Dynamic Execution Paths}, where an agent's operational sequence emerges from non-deterministic reasoning, and \textbf{Cross-Boundary Interactions}, where agents spawn subprocesses (\texttt{bash}, \texttt{curl}, \texttt{git}) that escape the monitoring scope of the parent process. 

% The danger of the unbridged semantic gap is not only theoretical. It manifests in significant reliability and security risks, as demonstrated by these real-world failure scenarios. In an \textbf{Unintended System Modification} incident, an AI agent tasked with code review misinterprets its instructions and begins modifying the production codebase—the semantic gap is exposed when application-level tools see the benign \emph{intent} ("review code") but are completely blind to the destructive system \emph{actions} (\texttt{git commit} and file modifications) occurring in a subprocess. A \textbf{Costly Reasoning Loop} occurs when a data analysis agent enters an infinite cycle calling expensive LLM APIs to solve an impossible problem, consuming thousands of dollars—framework logs reveal the \emph{action} (high volume of API calls) but provide no insight into the semantic \emph{intent}, which is a looping pattern of thought, making it impossible to distinguish between productive work and a costly failure mode without bridging the gap. In a \textbf{Cross-Process Exploitation}, an agent compromised via prompt injection writes a malicious shell script to \texttt{/tmp} and executes it—the semantic gap here is critical as system monitors see the file-write and process-execution \emph{actions}, but without linking them to the malicious \emph{intent} from the prompt, they appear as benign, uncorrelated events, and only by bridging the gap can the full attack narrative be reconstructed. These incidents exemplify a new threat model unique to AI agents, where the disconnect between intent and action can lead to prompt injection attacks, goal drift, and unintended capability escalation.

% Because of this gap, observing only the intent or only the action is insufficient and misleading, demanding a new approach that can holistically link the two.

% \textbf{SDK-Based and Proxy-Based Instrumentation} solutions like LangSmith~\cite{langsmith}, Langfuse~\cite{langfuse}, and Helicone~\cite{helicone} operate on the \textbf{intent side} of the gap. They excel at capturing LLM prompts, responses, and tool choices. However, they are blind to any system \textbf{actions} that occur outside the instrumented framework, such as operations within a spawned shell script. Furthermore, their reliance on in-process hooks makes them brittle against rapid framework evolution and assumes a cooperative agent that will not be compromised to bypass logging. Noble et al.~\cite{Noble2025IBM} further illustrate these limitations, noting SDK-based observability's inability to reliably capture real-time malicious intent or system-level anomalies.

% \textbf{Generic System-Level Monitoring} tools like Falco~\cite{falco} and Tracee~\cite{tracee} operate on the \textbf{action side}. They provide comprehensive visibility into every system call and process execution but lack all semantic context. To these tools, an agent writing a file is just a file write; they cannot determine the \textbf{intent} behind it. As demonstrated in the exploitation scenario, they can report that a script was executed from \texttt{/tmp}, but cannot link this action back to the agent's reasoning, failing to distinguish a malicious attack from a legitimate task.

% Furthermore, reasoning-level observability remains underexplored. Rombaut et al.~\cite{Rombaut2025Watson} highlight the critical value of cognitive observability, where agent reasoning traces and implicit chains-of-thought are reconstructed for debugging, an aspect typically missed by conventional monitoring.

% No existing solution can simultaneously capture semantic intent and system actions, maintain visibility across process boundaries, and remain resilient to framework changes. This critical failure to bridge the semantic gap motivates our novel approach.


\begin{center}
\begin{Verbatim}[fontsize=\small, commandchars=\\\{\}]
┌─────────────────────────────────────────────────┐
│             System Environment                  │
│  (Operating System, Containers, Services)       │
│                                                 │
│  ┌─────────────────────────────────────────┐   │
│  │      Agent Runtime Framework            │   │  ← Application Layer
│  │   (LangChain, AutoGen, Claude Code)     │   │
│  │   • Prompt orchestration                │   │
│  │   • Tool execution logic                │   │
│  │   • State management                    │   │
│  └─────────────────────────────────────────┘   │
│                    ↕                            │
│  ═══════════════════════════════════════════   │  ← Network Boundary
│           (TLS-encrypted traffic)               │     (Observable)
│                    ↕                            │
│  ┌─────────────────────────────────────────┐   │
│  │         LLM Service Provider            │   │
│  │    (OpenAI API, Local Models)           │   │
│  └─────────────────────────────────────────┘   │
│                                                 │
│  ═══════════════════════════════════════════   │  ← Kernel Boundary
│         (System calls, File I/O)                │     (Observable)
└─────────────────────────────────────────────────┘
\end{Verbatim}
\end{center}

Of course. Here are the restructured and improved `Design` and `Implementation` sections, based on the previous analysis. The changes focus on sharpening the narrative, replacing the ASCII diagram with a proper figure description, and adding technical depth.

***

### Rewritten Section 4: Design

\section{Design}
The design of AgentSight is guided by a single imperative: to bridge the semantic gap between an agent's intent and its actions. We achieve this through a novel observability paradigm, \emph{boundary tracing}, which is built on a foundation of stable system interfaces and realized through a multi-signal correlation engine.

\subsection{Boundary Tracing: A Principled Approach}
Our approach is rooted in a key insight: while agent frameworks and internal logic are volatile, the system boundaries they must cross to perform any meaningful action are stable and unavoidable. Boundary tracing leverages this stability to provide durable and comprehensive observability. The primary goal of this approach is to enable **Semantic Correlation**, the ability to causally link high-level intentions with low-level system events. This goal is made possible by two foundational principles:

\begin{itemize}
    \item \textbf{Comprehensiveness:} By monitoring at the kernel, we ensure that no system-level action—from process creation to file I/O—can go unobserved, regardless of the agent's implementation language or its attempts to evade monitoring by spawning subprocesses.
    \item \textbf{Stability:} System call ABIs and network protocols evolve far more slowly than agent frameworks. Monitoring at these stable interfaces provides a durable solution that is resilient to the constant, breaking changes common in agent libraries.
\end{itemize}
This paradigm fundamentally shifts the trust model from assuming a cooperative agent to enforcing observation at tamper-proof system boundaries.

\subsection{System Architecture: Observing the Boundaries}
AgentSight's architecture is designed to simultaneously tap into the two critical boundaries an agent interacts with: the network boundary for semantic intent and the kernel boundary for system actions. Figure 1 illustrates this architecture. We use eBPF to place non-intrusive probes at both boundaries. Probes on SSL library functions in userspace capture the decrypted **Intent Stream** (LLM prompts and responses), while probes at the kernel level capture the **Action Stream** (syscalls, process events). Both streams are processed by our userspace correlation engine, which joins them to produce a unified, causally-linked event trace.

\begin{figure}[h!]
    \centering
    % It is highly recommended to create this diagram using a tool like TikZ, Inkscape, or another vector graphics editor for the final paper.
    % The following is a LaTeX description of the recommended figure.
    \includegraphics[width=\columnwidth]{placeholder_diagram.png} % Replace with your actual figure file
    \caption{\textbf{AgentSight System Architecture.} The agent process communicates with LLM services across the Network Boundary and interacts with the OS via the Kernel Boundary. eBPF uprobes intercept decrypted TLS traffic within the agent's userspace to capture the semantic \textbf{Intent Stream}. Simultaneously, eBPF kprobes and tracepoints monitor kernel activity to capture the \textbf{Action Stream}. Both streams are fed into the AgentSight Correlation Engine, which causally links them to produce a complete, contextualized event trace.}
    \label{fig:architecture}
\end{figure}

\subsection{Bridging the Gap: Core Design Decisions}
Several key decisions enable AgentSight to effectively bridge the semantic gap:

\textbf{eBPF for Safe, Unified Probing:} We chose eBPF because it provides a single, production-safe technology to access both userspace and kernel data streams. Its verified safety model eliminates the risks of kernel modules, and its performance is vastly superior to traditional userspace hooking or `ptrace`-based approaches.

\textbf{Uprobes for Transparent TLS Interception:} To capture semantic intent, we intercept decrypted data directly from the agent's memory by placing `uprobes` on SSL library functions (e.g., `SSL_read`/`SSL_write`). This approach is superior to network-level packet capture, as it avoids the complexity of TLS key management, and more efficient than proxy-based solutions, which introduce network latency and configuration overhead.

\textbf{Multi-Signal Causal Correlation Engine:} The core of our design is a correlation strategy that establishes causality between intent and action. We designed a multi-signal engine that relies on:
\begin{enumerate}
    \item \textbf{Process Lineage:} Building a complete process tree by tracking `fork` and `execve` events to link actions in child processes back to the parent agent.
    \item \textbf{Temporal Proximity:} Associating actions that occur within a narrow time window immediately following an LLM response.
    \item \textbf{Argument Matching:} Directly matching content from LLM responses—such as filenames, URLs, or commands—with the arguments of subsequent system calls.
\end{enumerate}

---

### Rewritten Section 5: Implementation

\section{Implementation}
AgentSight is implemented as a userspace daemon written in Rust that orchestrates a suite of eBPF programs. The system is designed for high performance and low overhead, processing raw event streams from the kernel to produce correlated, human-readable observability data.

\subsection{Data Collection at the Boundaries}
Our eBPF probes are responsible for capturing the raw intent and action streams from the system.

\textbf{Capturing Semantic Intent (TLS):} An eBPF program utilizing `uprobes` is attached to `SSL_read` and `SSL_write` functions in dynamically linked crypto libraries (e.g., OpenSSL, BoringSSL). This allows us to intercept all decrypted LLM communications. A significant challenge here is handling streaming protocols like Server-Sent Events (SSE), which fragment a single JSON response across numerous `SSL_read` calls. Our userspace daemon implements a stateful reassembly mechanism that buffers data chunks per-connection and parses them for event boundaries (`\n\n`) to reconstruct complete messages.

\textbf{Capturing System Actions (Kernel):} A second eBPF program monitors kernel activity. We use efficient, stable `tracepoints` like `sched_process_fork` and `sched_process_exit` to build a process tree. For detailed actions, we use `kprobes` to dynamically attach to specific system calls relevant to agent behavior, such as `openat2` (file access), `connect` (network connections), and `execve` (program execution). Aggressive in-kernel filtering ensures that only events originating from targeted agent processes are sent to userspace, minimizing overhead.

\subsection{The Correlation Engine}
The Rust-based userspace daemon houses the core correlation engine. It consumes the high-throughput intent and action streams from eBPF ring buffers and performs three stages of analysis:

\begin{enumerate}
    \item \textbf{Enrichment:} Raw events are enriched with context. For example, a file descriptor from an `openat2` call is mapped to a full file path, and a process ID is mapped to its full command line and parent process from the process tree.
    \item \textbf{Stateful Tracking:} The engine maintains the state of each agent and its children in a process-tree data structure. It also tracks open files and network connections for each process.
    \item \textbf{Causal Linking:} Using the enriched data and maintained state, the engine applies the multi-signal correlation logic described in our design. It searches for system actions that match the "fingerprint" of a preceding LLM intent based on process lineage, temporal proximity (within a 100-500ms window), and content matching, thereby bridging the semantic gap and producing a single, coherent event trace.
\end{enumerate}
This entire pipeline is designed for streaming analysis, avoiding the need to store complete event histories and enabling real-time detection with a memory footprint of less than 200MB on a typical system. 

\section{Evaluation}

\subsection{Experimental Setup}

We evaluated AgentSight on AWS EC2 c5.2xlarge instances (8 vCPUs, 16GB RAM, Linux 5.15) across three agent frameworks: LangChain 0.1.0, AutoGen 0.2.0, and Claude Code. Workloads included code generation (simple functions to complex systems), data analysis (CSV/JSON processing), and system administration (package management). We measured end-to-end completion time, CPU/memory usage, and event rates, running each experiment 50 times with and without AgentSight to ensure statistical significance.

\subsection{Performance Evaluation}


\begin{table}[h]
\centering
\caption{Performance Overhead of AgentSight}
\label{tab:performance}
\begin{tabular}{lrrrr}
\toprule
Workload Type & Baseline & With AgentSight & Overhead & Events/sec \\
\midrule
Code Generation (simple) & 12.3s & 12.5s & 1.6\% & 432 \\
Code Generation (complex) & 87.2s & 89.1s & 2.2\% & 1,247 \\
Data Analysis & 34.5s & 35.2s & 2.0\% & 892 \\
System Admin Tasks & 23.1s & 23.7s & 2.6\% & 2,156 \\
Idle Agent & 0.1\% CPU & 0.3\% CPU & +0.2\% & 12 \\
\bottomrule
\end{tabular}
\end{table}

Table~\ref{tab:performance} summarizes performance impacts across different agent task types.The results consistent sub-3\% overhead across all workloads. Simple code generation shows minimal 1.6\% impact due to fewer system events, while complex generation with extensive file operations reaches 2.2\%. System administration tasks exhibit the highest overhead at 2.6\% due to intensive process spawning and file system operations, generating over 2,000 events per second. Even under peak load, AgentSight maintains responsive performance with fixed memory footprint (192MB on 8-core systems) and negligible idle overhead (0.2\% CPU), validating our kernel-space filtering design.

\subsection{Effectiveness Evaluation}

We evaluated AgentSight's effectiveness through three comprehensive case studies that demonstrate its ability to detect security threats, identify performance issues, and provide insights into complex multi-agent systems.

\subsection{Case Studies}

\subsubsection{Case Study 1: Detecting Prompt Injection Attacks}

We tested AgentSight's ability to detect prompt injection attacks where a data analysis agent received a crafted prompt embedding malicious commands within a legitimate request to analyze sales data, ultimately exfiltrating /etc/passwd. AgentSight captured the complete attack chain: LLM interaction with suspicious prompt (T+0ms), agent-generated Python script with embedded curl command (T+125ms), subprocess spawn (T+342ms), outbound HTTPS connection to suspicious domain (T+367ms), and sensitive file read (T+368ms). The correlation engine identified potential data exfiltration with 92\% confidence by linking the prompt injection with subsequent file access, suspicious network connection, and 1.2KB transfer matching the file size. This detection proves critical for production deployments where traditional application-level monitoring would miss the correlation between initial prompts and system activities across process boundaries, while AgentSight's boundary tracing captures the complete attack narrative enabling rapid incident response.

\subsubsection{Case Study 2: Reasoning Loop Detection}

An agent attempting a complex task entered an infinite reasoning loop with circular dependencies where solving X required solving Y, but solving Y required solving X—a pattern common when agents encounter problems outside their training distribution. AgentSight detected this through multiple mechanisms: pattern analysis tracking LLM API call sequences with semantic similarity metrics to identify repeated prompt structures even when rephrased; resource monitoring showing constant token consumption without progress markers instead of healthy decreasing usage; temporal correlation revealing suspiciously regular intervals between API calls characteristic of stuck retry logic; and semantic progress tracking using embedding-based similarity to detect reasoning stagnation. The system triggered an alert after detecting three complete cycles where the agent had consumed 4,800 tokens across 12 API calls, with AgentSight's intervention saving an estimated \$2.40 in API costs and preventing service degradation—demonstrating the critical importance of semantic-aware monitoring for autonomous agents.

\subsubsection{Case Study 3: Multi-Agent Coordination Monitoring}

AgentSight monitored three agents collaborating on software development (Agent A: architecture design, Agent B: implementation, Agent C: testing), capturing 12,847 total events with 342 correlated actions across 27 synchronization points involving 15 shared files and 3 network endpoints. The analysis revealed critical inefficiencies invisible to traditional monitoring: Agent B spent 34\% of time blocked on Agent A's multiple design revisions triggering cascading rework; file locking patterns showed resource contention with Agent C's testing conflicting with Agent B's implementation causing 23 retry cycles; inter-agent communication through shared files generated 1,800 unnecessary file system operations from 2-second polling intervals; yet agents developed emergent coordination with Agent B learning to batch changes, reducing test executions by 40\%. These insights demonstrate that explicit coordination mechanisms could reduce runtime by 25\% and message-based communication would eliminate 90\% of polling overhead—revealing how boundary tracing uniquely captures multi-agent system dynamics that application-level monitoring cannot observe across process boundaries.

% \section{Future Work}
% While AgentSight establishes the power of boundary tracing, our work opens several avenues for future research focused on advancing the safety and observability of autonomous systems.

% First, we plan to enhance our causal correlation engine with machine learning. Currently, AgentSight uses a rule-based approach to bridge the semantic gap. Future work will involve training models on correlated intent-action traces to automatically detect novel anomalous behaviors, moving from identifying known attack patterns to discovering unknown unknowns. This includes using NLP to assess the semantic risk of prompts before they are even executed.

% Second, we will extend our framework from passive observation to active intervention. By integrating formal specification languages, an operator could define safety policies (e.g., "this agent may not access files outside its working directory"). AgentSight could then act as a runtime monitor that enforces these policies, functioning as a "circuit breaker" to terminate harmful actions before they complete.

% Finally, we will address the challenges of scale and integration. This involves extending boundary tracing to distributed environments to monitor multi-node agents and integrating with standard formats like OpenTelemetry to ensure AgentSight's insights are available in existing observability platforms. This also includes exploring privacy-preserving techniques to enable analysis without exposing sensitive data in prompts or system interactions.

\section{Conclusion}

This paper introduced boundary tracing as a novel observability paradigm for AI agents, monitoring at stable system interfaces rather than within rapidly evolving application code. AgentSight demonstrates this approach's feasibility, achieving sub-3\% overhead while detecting prompt injection attacks and identifying reasoning loops before resource exhaustion. By combining TLS interception with eBPF-based kernel monitoring, we bridge the semantic gap between agent intentions and system effects. We release AgentSight as open source to address the critical challenge of safely deploying autonomous AI systems in production environments.

\textbf{Repository}: \url{https://github.com/eunomia-bpf/agentsight}

\bibliographystyle{ACM-Reference-Format}
\bibliography{ai}

\end{document}

\bibliographystyle{ACM-Reference-Format}
\bibliography{ai}
\end{document}