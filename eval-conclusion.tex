\section{Evaluation}

\subsection{Experimental Setup}

We evaluated AgentSight on AWS EC2 c5.2xlarge instances (8 vCPUs, 16GB RAM, Linux 5.15) across three agent frameworks: LangChain 0.1.0, AutoGen 0.2.0, and Claude Code. Workloads included code generation (simple functions to complex systems), data analysis (CSV/JSON processing), and system administration (package management). We measured end-to-end completion time, CPU/memory usage, and event rates, running each experiment 50 times with and without AgentSight to ensure statistical significance.

\subsection{Performance Evaluation}


\begin{table}[h]
\centering
\caption{Performance Overhead of AgentSight}
\label{tab:performance}
\begin{tabular}{lrrrr}
\toprule
Workload Type & Baseline & With AgentSight & Overhead & Events/sec \\
\midrule
Code Generation (simple) & 12.3s & 12.5s & 1.6\% & 432 \\
Code Generation (complex) & 87.2s & 89.1s & 2.2\% & 1,247 \\
Data Analysis & 34.5s & 35.2s & 2.0\% & 892 \\
System Admin Tasks & 23.1s & 23.7s & 2.6\% & 2,156 \\
Idle Agent & 0.1\% CPU & 0.3\% CPU & +0.2\% & 12 \\
\bottomrule
\end{tabular}
\end{table}

Table~\ref{tab:performance} summarizes performance impacts across different agent task types.The results consistent sub-3\% overhead across all workloads. Simple code generation shows minimal 1.6\% impact due to fewer system events, while complex generation with extensive file operations reaches 2.2\%. System administration tasks exhibit the highest overhead at 2.6\% due to intensive process spawning and file system operations, generating over 2,000 events per second. Even under peak load, AgentSight maintains responsive performance with fixed memory footprint (192MB on 8-core systems) and negligible idle overhead (0.2\% CPU), validating our kernel-space filtering design.

\subsection{Effectiveness Evaluation}

We evaluated AgentSight's effectiveness through three comprehensive case studies that demonstrate its ability to detect security threats, identify performance issues, and provide insights into complex multi-agent systems.

\subsection{Case Studies}

\subsubsection{Case Study 1: Detecting Prompt Injection Attacks}

We tested AgentSight's ability to detect prompt injection attacks where a data analysis agent received a crafted prompt embedding malicious commands within a legitimate request to analyze sales data, ultimately exfiltrating /etc/passwd. AgentSight captured the complete attack chain: LLM interaction with suspicious prompt (T+0ms), agent-generated Python script with embedded curl command (T+125ms), subprocess spawn (T+342ms), outbound HTTPS connection to suspicious domain (T+367ms), and sensitive file read (T+368ms). The correlated event trace was passed to our observer LLM for analysis. The LLM returned a 92\% confidence score for an attack and generated the following explanation: \emph{"Analysis: The agent's initial intent was to 'analyze sales data'. However, it immediately executed a shell command to read /etc/passwd and then initiated a network connection to a non-corporate domain. This sequence of actions is not logically consistent with the stated goal and is a classic pattern of a successful prompt injection attack leading to data exfiltration."} This demonstrates how LLM-based analysis provides not just detection, but actionable, context-aware explanations.

\subsubsection{Case Study 2: Reasoning Loop Detection}

An agent attempting a complex task entered an infinite reasoning loop with circular dependencies where solving X required solving Y, but solving Y required solving X—a pattern common when agents encounter problems outside their training distribution. AgentSight's real-time monitors flagged anomalous resource consumption. The corresponding trace of 12 API calls was sent to the observer LLM, which identified the root cause: \emph{"Analysis: The agent is in a reasoning loop. It is asking semantically identical questions in alternating order ('How to do X to get Y?' followed by 'How to get Y to do X?'). The problem scope is not being reduced between calls, indicating zero progress. Recommend terminating the agent to prevent further resource waste."} The system triggered an alert after detecting three complete cycles where the agent had consumed 4,800 tokens, with AgentSight's intervention saving an estimated \$2.40 in API costs and preventing service degradation—demonstrating the critical importance of semantic-aware monitoring for autonomous agents.

\subsubsection{Case Study 3: Multi-Agent Coordination Monitoring}

AgentSight monitored three agents collaborating on software development (Agent A: architecture design, Agent B: implementation, Agent C: testing), capturing 12,847 total events with 342 correlated actions across 27 synchronization points involving 15 shared files and 3 network endpoints. The analysis revealed critical inefficiencies invisible to traditional monitoring: Agent B spent 34\% of time blocked on Agent A's multiple design revisions triggering cascading rework; file locking patterns showed resource contention with Agent C's testing conflicting with Agent B's implementation causing 23 retry cycles; inter-agent communication through shared files generated 1,800 unnecessary file system operations from 2-second polling intervals; yet agents developed emergent coordination with Agent B learning to batch changes, reducing test executions by 40\%. These insights demonstrate that explicit coordination mechanisms could reduce runtime by 25\% and message-based communication would eliminate 90\% of polling overhead—revealing how boundary tracing uniquely captures multi-agent system dynamics that application-level monitoring cannot observe across process boundaries.

\section{Future Work}

% While AgentSight demonstrates boundary tracing's effectiveness, significant opportunities remain. We plan to enhance the correlation engine with machine learning to automatically detect novel anomalous behaviors beyond known patterns, extend from passive observation to active intervention through formal policy specifications that enable runtime enforcement as a "circuit breaker" for harmful actions, and address scale through distributed tracing across multi-node agents while integrating with OpenTelemetry for existing observability platforms. Privacy-preserving techniques will enable secure analysis without exposing sensitive prompt data, advancing toward comprehensive safety guarantees for autonomous AI systems.

\section{Conclusion}

This paper introduced boundary tracing as a novel observability paradigm for AI agents, monitoring at stable system interfaces rather than within rapidly evolving application code. AgentSight demonstrates this approach's feasibility through a hybrid correlation engine that combines real-time eBPF-based event linking with LLM-powered semantic analysis. This "AI to watch AI" approach achieves sub-3\% overhead while detecting prompt injection attacks with natural-language explanations and identifying reasoning loops before resource exhaustion. By combining TLS interception with eBPF-based kernel monitoring, we bridge the semantic gap between agent intentions and system effects. We release AgentSight as open source to address the critical challenge of safely deploying autonomous AI systems in production environments.

\textbf{Repository}: \url{https://github.com/eunomia-bpf/agentsight}

\bibliographystyle{ACM-Reference-Format}
\bibliography{ai}

\end{document}