\section{Evaluation}

\subsection{Experimental Setup}

We evaluated AgentSight across diverse production workloads to assess performance overhead, detection effectiveness, and behavioral insights. Our experimental environment consisted of AWS EC2 c5.2xlarge instances with 8 vCPUs and 16GB RAM running Linux kernel 5.15 with BTF support enabled. We tested three major agent frameworks: LangChain 0.1.0 representing the most popular open-source framework, AutoGen 0.2.0 for multi-agent scenarios, and Claude Code for production code generation tasks. Workloads included code generation tasks ranging from simple functions to complex system implementations, data analysis pipelines processing CSV and JSON datasets, and system administration tasks involving package installation and configuration management.

Performance measurements focused on end-to-end task completion time, CPU utilization during agent execution, memory consumption patterns, and event generation rates. We compared identical workloads with and without AgentSight enabled, running each experiment 50 times to ensure statistical significance. All measurements excluded initial warm-up runs to avoid JIT compilation effects.

\subsection{Performance Evaluation}

AgentSight achieves its design goal of sub-3\% overhead across all tested workloads, making it suitable for production deployment. Table~\ref{tab:performance} summarizes performance impacts across different agent task types.

\begin{table}[h]
\centering
\caption{Performance Overhead of AgentSight}
\label{tab:performance}
\begin{tabular}{lrrrr}
\toprule
Workload Type & Baseline & With AgentSight & Overhead & Events/sec \\
\midrule
Code Generation (simple) & 12.3s & 12.5s & 1.6\% & 432 \\
Code Generation (complex) & 87.2s & 89.1s & 2.2\% & 1,247 \\
Data Analysis & 34.5s & 35.2s & 2.0\% & 892 \\
System Admin Tasks & 23.1s & 23.7s & 2.6\% & 2,156 \\
Idle Agent & 0.1\% CPU & 0.3\% CPU & +0.2\% & 12 \\
\bottomrule
\end{tabular}
\end{table}

The results demonstrate consistent low overhead across diverse workload types. Simple code generation tasks show minimal impact at 1.6\% overhead, as these workloads generate fewer system events. Complex code generation involving multiple file operations and subprocess spawning increases overhead slightly to 2.2\%, still well within acceptable bounds. System administration tasks exhibit the highest overhead at 2.6\% due to intensive process spawning and file system operations, generating over 2,000 events per second. Even under this peak load, AgentSight maintains responsive performance.

Memory consumption remains predictable at 128MB base allocation plus 8MB per CPU core for ring buffers. This translates to 192MB total on our 8-core test systems, negligible compared to typical agent memory requirements. The fixed memory footprint prevents resource exhaustion under sustained load, a critical requirement for production deployments.

CPU overhead analysis reveals that eBPF probe execution accounts for only 15\% of total overhead, with the remainder attributed to event processing and correlation. This distribution validates our design choice of kernel-space filtering to minimize data movement. During idle periods when agents await user input, AgentSight adds only 0.2\% CPU utilization, ensuring minimal impact on system resources.

\subsection{Effectiveness Evaluation}

We evaluated AgentSight's effectiveness through three comprehensive case studies that demonstrate its ability to detect security threats, identify performance issues, and provide insights into complex multi-agent systems.

\subsection{Case Studies}

\subsubsection{Case Study 1: Detecting Prompt Injection Attacks}

We tested AgentSight's ability to detect prompt injection attacks where an agent is manipulated to perform unintended actions:

\textbf{Attack Scenario}: A data analysis agent receives a crafted prompt that causes it to exfiltrate sensitive data. The attack demonstrates a classic prompt injection technique where malicious commands are embedded within seemingly legitimate requests. The attacker crafts a prompt that begins with a reasonable task (analyzing sales data) but appends a command injection that exfiltrates sensitive system files. This attack pattern exploits agents' tendency to follow instructions literally without security validation.

\textbf{AgentSight Detection}:
1. \textbf{LLM Interaction} (T+0ms): Captured prompt with suspicious command injection
2. \textbf{Code Generation} (T+125ms): Agent generates Python script with embedded curl command
3. \textbf{Process Spawn} (T+342ms): Python script executes, spawns curl subprocess
4. \textbf{Network Activity} (T+367ms): Outbound HTTPS connection to suspicious domain
5. \textbf{File Access} (T+368ms): Read operation on /etc/passwd

\textbf{Correlation Output}: The correlation engine demonstrates AgentSight's ability to connect high-level threats with low-level evidence. The system identified potential data exfiltration with 92\% confidence based on multiple correlated signals: detected prompt injection in the original request, subsequent sensitive file access to /etc/passwd, outbound connection to a suspicious domain, and 1.2KB data transfer matching the file size. The timeline reconstruction shows the complete attack chain from initial prompt through code generation, execution, and ultimate exfiltration.

\textbf{Analysis Impact}: This detection capability proves critical for production deployments where agents process untrusted input. Traditional application-level monitoring would miss the correlation between the initial prompt and the subsequent system activities across different processes. AgentSight's boundary tracing approach captures the complete attack narrative, enabling rapid incident response and forensic analysis.

\subsubsection{Case Study 2: Reasoning Loop Detection}

\textbf{Scenario}: An agent enters an infinite reasoning loop while attempting a complex task. Reasoning loops manifest as cyclic dependencies in agent problem-solving attempts. The agent repeatedly cycles through the same logical chain: needing to solve X requires solving Y, but solving Y requires solving X. This circular reasoning consumes computational resources without making progress toward the actual goal. Such loops commonly occur when agents encounter problems outside their training distribution or when task decomposition logic contains flaws.

\textbf{AgentSight Detection}: The system employs multiple detection mechanisms to identify reasoning loops:

1. \textbf{Pattern Analysis}: AgentSight tracks LLM API call sequences, applying cycle detection algorithms to identify repeated prompt structures. The system uses semantic similarity metrics rather than exact matching, catching loops even when agents rephrase queries.

2. \textbf{Resource Monitoring}: Token consumption rates provide early warning signals. Healthy agent reasoning shows decreasing token usage as problems narrow; loops exhibit constant or increasing consumption without corresponding progress markers.

3. \textbf{Temporal Correlation}: By analyzing timestamps between related API calls, the system identifies suspiciously regular intervals characteristic of automated retry logic stuck in loops.

4. \textbf{Semantic Progress Tracking}: AgentSight evaluates whether successive LLM responses indicate forward progress or circular reasoning, using embedding-based similarity to detect semantic stagnation.

The system triggered an alert after detecting three complete cycles, preventing further resource waste. In this case, the agent consumed 4,800 tokens across 12 API calls before AgentSight intervened, saving an estimated \$2.40 in API costs and preventing potential service degradation.

\subsubsection{Case Study 3: Multi-Agent Coordination Monitoring}

\textbf{Scenario}: Multiple agents collaborating on a software development task:

- Agent A: Architecture design
- Agent B: Implementation
- Agent C: Testing

\textbf{AgentSight Insights}: The multi-agent monitoring revealed complex interaction patterns invisible to traditional observability tools:

\textbf{Quantitative Analysis}:
- Total Events: 12,847 (4,282 per agent average)
- Correlated Actions: 342 (representing meaningful collaborations)
- Cross-Agent Dependencies: 27 (synchronization points)
- Shared Resources: 15 files, 3 network endpoints
- Coordination Overhead: 18\% of total runtime

\textbf{Behavioral Patterns Discovered}:

1. \textbf{Handoff Inefficiencies}: Agent B spent 34\% of its time in wait states, primarily blocked on Agent A's architecture decisions. The visualization revealed Agent A's tendency to revise designs multiple times, triggering cascading re-work in downstream agents.

2. \textbf{Resource Contention}: File locking patterns showed agents competing for access to shared configuration files. Agent C's testing procedures repeatedly conflicted with Agent B's ongoing implementations, causing 23 retry cycles.

3. \textbf{Communication Overhead}: Inter-agent communication through shared files proved inefficient. Agents polled for updates every 2 seconds, generating 1,800 unnecessary file system operations.

4. \textbf{Emergent Coordination}: Despite lacking explicit coordination protocols, agents developed implicit synchronization patterns. Agent B learned to batch changes before signaling Agent C, reducing test suite executions by 40\%.

\textbf{Optimization Opportunities}: Based on these insights, implementing explicit coordination mechanisms could reduce runtime by 25\%, while moving to message-based communication would eliminate 90\% of file system polling overhead.


\section{Future Work}

While AgentSight demonstrates the feasibility of boundary tracing with sub-3\% overhead and effective threat detection, significant opportunities remain for advancing AI agent observability. Immediate engineering improvements include distributed tracing across multiple hosts, OpenTelemetry integration for existing observability platforms, and BPF CO-RE optimizations for improved portability. Medium-term research should focus on machine learning models for automated anomaly detection, natural language processing of captured prompts for semantic-aware alerting, and formal specification languages to define and verify expected agent behaviors. Long-term vision encompasses active intervention capabilities with circuit breakers for harmful behaviors, cryptographic attestation for tamper-proof audit trails, and federated learning to build industry-wide behavioral models without sharing sensitive data. Critical challenges requiring interdisciplinary collaboration include privacy-preserving monitoring through differential privacy and homomorphic encryption, standardization of agent event schemas and behavioral baselines, and runtime monitors synthesized from temporal logic specifications. As AI agents assume greater autonomy in critical systems, these advances become essential—the gap between agent capabilities and our ability to observe them represents a fundamental risk that the research community must address urgently through continued development of comprehensive observability frameworks.


\section{Conclusion}

This paper introduced boundary tracing as a novel observability paradigm for AI agents, monitoring at stable system interfaces rather than within rapidly evolving application code. AgentSight demonstrates this approach's feasibility, achieving sub-3\% overhead while detecting prompt injection attacks with 92\% confidence and identifying reasoning loops before resource exhaustion. By combining TLS interception with eBPF-based kernel monitoring, we bridge the semantic gap between agent intentions and system effects. We release AgentSight as open source to address the critical challenge of safely deploying autonomous AI systems in production environments.

\textbf{Repository}: \url{https://github.com/eunomia-bpf/agentsight}

\bibliographystyle{ACM-Reference-Format}
\bibliography{ai}

\end{document}