
\documentclass[sigplan,screen,review,9pt]{acmart}
\settopmatter{printacmref=false, printccs=false, printfolios=false}
\renewcommand\footnotetextcopyrightpermission[1]{}

\usepackage{listings}     % For ASCII-art / code blocks
\usepackage{booktabs}     % Nicer tables
\usepackage{array}        % Column types
\usepackage{tabularx}     % Automatic column width
\usepackage{enumitem}     % Compact lists

\usepackage{newunicodechar} % 自定义 Unicode 字符
\usepackage{fancyvrb}   % 增强的 verbatim 环境



% 定义方框字符(可选,确保正确渲染)
\newunicodechar{┌}{\symbol{"250C}} % 上左角
\newunicodechar{┐}{\symbol{"2510}} % 上右角
\newunicodechar{└}{\symbol{"2514}} % 下左角
\newunicodechar{┘}{\symbol{"2518}} % 下右角
\newunicodechar{─}{\symbol{"2500}} % 横线
\newunicodechar{│}{\symbol{"2502}} % 竖线
\newunicodechar{☁}{{\fontspec{Symbola}\symbol{"2601}}} % 云朵符号

\begin{document}

\title{From Fragmentation to Cognition: A Vision for AI Agent Observability}


\author{}


\sloppy
\begin{abstract}
AI agent systems are transforming software infrastructure, yet their large-scale deployment faces critical challenges: non-deterministic failures from complex reasoning, prohibitive costs across multi-layer execution stacks, and severe fragmentation from heterogeneous vendor ecosystems. Current observability approaches—traditional APM, model-centric LLM monitoring, and framework-specific instrumentation—fail to address these challenges, creating brittle, insecure monitoring with isolated SDK silos. We argue that scalable agent observability requires a fundamental architectural shift: (1) a unified, zero-instrumentation Data Plane that captures semantics at stable system boundaries across all software layers, and (2) an autonomous Cognitive Plane where specialized agents observe and manage other agents. This vision addresses the unique demands of agent systems while providing a foundation for secure, cost-effective, and semantically-aware observability at scale.
\end{abstract}


\maketitle



\section{Introduction: The Imperative for Large-Scale Agent Observability}

\subsection{The Current State of AI Agent Deployment}

The rise of AI-powered agentic systems is transforming modern software infrastructure~\cite{wang2024survey,guo2024survey}. Frameworks like AutoGen~\cite{autogen}, LangChain~\cite{langchain}, Claude Code~\cite{claudecode}, and gemini-cli~\cite{geminicli} orchestrate large language models (LLMs) to automate software engineering tasks, data analysis pipelines, and multi-agent decision-making~\cite{tran2025survey}. However, their large-scale production deployment faces three fundamental challenges that existing observability paradigms cannot address:

\textbf{1. Non-determinism and Complexity Lead to Security Risks.} Unlike traditional software components that produce deterministic, easily observable behaviors, AI-agent systems generate open-ended, non-deterministic outputs, often conditioned on hidden internal states and emergent interactions between multiple agents. This creates novel failure modes: wrong facts, infinite reasoning loops~\cite{zhang2024breakingagents}, forgotten instructions, and emergent mis-coordination between agents. More critically, agents can be compromised via prompt injection attacks~\cite{indirect-prompt-inject}, potentially evading their own logging mechanisms to hide malicious behavior. Traditional crash dumps and error logs are insufficient—we need semantic traces of \emph{why} an agent made a decision, not just \emph{what} it executed.

\textbf{2. Multi-Layer Execution Stacks Incur Prohibitive Costs.} Agent systems span multiple expensive layers: LLM inference (token generation), reasoning frameworks (chain-of-thought processing), tool execution (API calls, subprocess spawning), and persistent memory (vector databases). Each layer contributes to operational costs, yet current observability solutions provide no unified view to optimize spending. Cost overruns from runaway reasoning loops, redundant API calls, or inefficient tool usage remain invisible until billing alerts trigger—far too late for real-time intervention.

\textbf{3. Extreme Fragmentation Across a Multi-Vendor Ecosystem.} Modern agent deployments are radically decentralized: LLM serving is managed by SaaS providers (OpenAI, Anthropic), agent logic by application developers, execution environments by infrastructure teams, and tools (e.g., MCP servers) by third-party vendors. Each stakeholder instruments their layer independently, creating a fragmented observability landscape of incompatible SDKs, proprietary formats, and security vulnerabilities. This "SDK hell" makes end-to-end debugging nearly impossible and introduces attack surfaces at every integration point.

\begin{table*}[t]
  \caption{How AI-agent observability differs from classic software observability}
  \label{tab:diff}
  \begin{tabularx}{\linewidth}{@{}>{\raggedright\arraybackslash}p{2.85cm}X X@{}}
    \toprule
    \textbf{Dimension} &
    \textbf{Traditional app / micro-service} &
    \textbf{LLM or multi-agent system} \\
    \midrule
    What you try to “see” &
    Latency, errors, CPU, GC, SQL counts, request paths &
    \emph{Semantics}—prompt / tool trace, reasoning steps, toxicity, hallucination rate, persona drift, tokens/\$ spent \\
    Ground truth &
    Deterministic spec: given~$X$ produce~$Y$ or raise exception &
    Open-ended output: many acceptable~$Y$; quality judged by similarity, helpfulness, policy compliance \\
    Failure modes &
    Crashes, 5xx, memory leaks, deadlocks &
    Wrong facts, infinite loops, forgotten instructions, emergent mis-coordination between agents \\
    Time scale &
    Millisecond spans; state usually dies at request end &
    Dialogue and scratch memories persist for hours/days; “state’’ hides in vector DB rows and system prompts \\
    Signal source &
    Structured logs and metrics emitted on purpose &
    Often \emph{inside plain-text TLS payloads}; plus tool-execution logs \\
    Fix workflow &
    Reproduce, attach debugger, patch code &
    Re-prompt, fine-tune, change tool wiring, tweak guardrails—code may be fine but the “thought process’’ is wrong \\
    Safety / audit &
    Trace shows what code ran &
    Need evidence of \emph{why} the model said something for compliance / incident reviews \\
    \bottomrule
  \end{tabularx}
\end{table*}

\subsection{Defining the Domain: What Agent Observability Is NOT}

To understand what scalable agent observability requires, we must first distinguish it from three related but fundamentally different domains:

\textbf{Not Traditional APM.} Classic application performance monitoring (APM) tools like Datadog and New Relic are designed for deterministic systems with well-defined failure modes (crashes, 5xx errors, memory leaks). AI agents, by contrast, fail \emph{semantically}—they produce plausible but incorrect outputs, enter infinite reasoning loops, or exhibit persona drift. Traditional metrics (CPU, latency, error rates) cannot capture these failure modes. We need semantic observability: tracking prompt evolution, tool interaction patterns, and the causal chain from user intent to agent decision.

\textbf{Not LLM-Centric Monitoring.} Existing LLM monitoring solutions (e.g., prompt safety filters, hallucination detectors) focus on single-turn model interactions and are inherently \emph{model-centric}. They operate at the inference endpoint, monitoring token generation and output quality. Agent observability, however, must be \emph{process-centric}, tracking multi-step reasoning trajectories, tool orchestration, cross-agent coordination, and long-lived state (vector databases, scratch memory). The observability scope extends far beyond the model boundary to encompass the entire autonomous decision-making process.

\textbf{Not LLM Serving Observability.} LLM serving platforms prioritize operational efficiency: throughput, latency percentiles, SLO compliance, and resource utilization (GPU, memory). This is infrastructure observability for the inference layer. Agent observability, conversely, must reason about \emph{behavioral correctness}: did the agent follow instructions, use tools appropriately, coordinate effectively with other agents, and achieve its goal within cost constraints? These are semantic properties orthogonal to serving metrics.

\subsection{Why the Difference Matters for Research}

\textbf{Instrumentation gap} – Agent logic and algorithm changes daily (new prompts, tools) or by itself at runtime. Relying on in-code hooks means constant churn; system-boundary tracing stays stable.

\textbf{Semantic telemetry} – We need new span attributes (``model.temp'', ``tool.role'', ``reasoning.loop\_id'') and new anomaly detectors (contradiction, persona drift).

\textbf{Causal fusion} – Research challenge: merge low-level events with high-level semantic spans into a single timeline so SREs can answer ``why my code is not work? what system is it run on and what command have you tried?''

\textbf{Tamper resistance} – If prompt-injection turns the agent malicious it may silence its own logs. Out-of-process and kernel level tracing provides an independent audit channel.

In short, AI-agent observability inherits the \textbf{unreliable, emergent behaviour} of AI Agents. Treat the agent runtime as a semi-trusted black box and observe at the system boundary: that's where the opportunities lie.

\section{Survey: The Current Observability Landscape}

\subsection{The Fragmentation Problem}

Current agent observability techniques rely predominantly on application-level instrumentation—callbacks, middleware hooks, or explicit logging—integrated within each agent framework. While intuitive, this approach suffers three fundamental limitations. First, agent frameworks evolve rapidly, changing prompts, tools, workflow and memory interfaces frequently. They can even modify their self code to create new tools, change prompts and behaviors. Thus, instrumentation embedded within agent codebases incurs significant maintenance overhead. Second, agent runtimes can be tampered with or compromised (e.g., via prompt injection), allowing attackers or buggy behaviors to evade logging entirely. Fourth, application-level instrumentation cannot reliably capture cross-agent semantics, such as reasoning loops, semantic contradictions, persona shifts, or the behaviors when it's interacting with it's environment, especially when interactions cross process or binary boundaries (e.g., external tools or subprocesses).

For security, consider a LLM agent first writing a bash file with malicious commands (not executed, safe), and then executing it with basic tool calls (often allowed). This requires system-wide observability and constraints.

\subsection{Industrial and Open-Source Solutions}

Below is a comprehensive landscape scan of LLM / AI-agent observability tooling as of 2025, focusing on offerings that (a) expose an SDK, proxy, or spec you can wire into an agent stack today and (b) ship some way to trace / evaluate / monitor model calls in production.

\begin{table*}[t]
\centering
\scriptsize
\begin{tabular}{p{0.5cm} p{2.2cm} p{2.3cm} p{2.8cm} p{1.8cm} p{2.5cm}}
\toprule
\# & Tool / SDK (year) & Integration path & What it provides & License / model & Notes \\
\midrule
1 & \textbf{LangSmith} (2023) & Add \texttt{import langsmith} to LangChain/LangGraph apps & Request/response traces, prompt \& token stats, evaluations & SaaS, free tier & Tight LangChain integration; OTel export beta \\
2 & \textbf{Helicone} (2023) & Reverse-proxy or Python/JS SDK & Logs OpenAI-style calls, cost/latency dashboards & OSS (MIT) + hosted & Proxy model requires no code changes \\
3 & \textbf{Traceloop} (2024) & One-line SDK import → OTel & OTel spans for prompts, tools, sub-calls & SaaS, free tier & Standard OTel data compatibility \\
4 & \textbf{Arize Phoenix} (2024) & \texttt{pip install}, OpenInference tracer & Local UI + vector store for traces, automatic evals & Apache-2.0 & Includes open-source UI for debugging \\
5 & \textbf{Langfuse} (2024) & SDK or OTel OTLP & Nested traces, cost metrics, prompt management & OSS (MIT) + cloud & Popular for RAG/multi-agent projects \\
6 & \textbf{WhyLabs LangKit} (2023) & Text metrics wrapper & Drift, toxicity, sentiment, PII detection & Apache-2.0 core & Focuses on text-quality metrics \\
7 & \textbf{PromptLayer} (2022) & Decorator or proxy & Prompt chain timeline, diff \& replay & SaaS & Early solution, minimal code changes \\
8 & \textbf{Literal AI} (2024) & Python SDK + UI & RAG-aware traces, eval experiments & OSS + SaaS & Targets chatbot product teams \\
9 & \textbf{W\&B Weave/Traces} (2024) & \texttt{import weave} or SDK & Links to W\&B projects, captures code/IO & SaaS & Integrates with existing W\&B workflows \\
10 & \textbf{Honeycomb Gen-AI} (2024) & Send OTel spans & Heat-maps on prompt spans, latency & SaaS & Built on mature trace store \\
11 & \textbf{OTel GenAI Conv.} (2024) & Spec + Python lib & Standard span names for models/agents & Apache-2.0 & Provides semantic conventions \\
12 & \textbf{OpenInference} (2023) & Tracer wrapper & JSON schema for traces & Apache-2.0 & Specification (not hosted service) \\
\bottomrule
\end{tabular}
\end{table*}

\subsection{Critique: The SDK Fragmentation Crisis}

Analysis of the current landscape reveals five critical insights:

\textbf{1. SDK Hell: Universal Application-Layer Dependence.} 11 of 12 surveyed tools require application-level hooks—wrapping function calls, proxying requests, or injecting middleware. This creates a brittle ecosystem: each framework update (LangChain, AutoGen, etc.) can break instrumentation, and agents that dynamically spawn tools or modify their own code bypass these wrappers entirely. The result is a perpetual maintenance burden and unreliable telemetry.

\textbf{2. OpenTelemetry Emergence but Incomplete Adoption.} Traceloop, Honeycomb, Langfuse, PromptLayer, and Phoenix (via OpenInference) have converged on OpenTelemetry (OTel) as the wire format~\cite{Liu2025OTel}, providing hope for standardization. However, semantic conventions remain immature—attributes like \texttt{model.temp}, \texttt{reasoning.loop\_id}, or \texttt{tool.coordination\_pattern} are not yet standardized, forcing each vendor to define proprietary extensions.

\textbf{3. Semantic Analysis Remains Nascent.} Only a subset (Phoenix, LangSmith, Langfuse, Literal AI) offer LLM-powered semantic evaluation (toxicity, hallucination detection, relevance scoring). Most solutions focus on operational metrics (latency, cost, throughput), leaving the critical "semantic gap"—understanding \emph{why} an agent made a decision—largely unaddressed.

\textbf{4. Zero System-Level Visibility.} No existing tool captures telemetry at the kernel or network boundary. None observe encrypted TLS traffic, \texttt{execve()} syscalls, or GPU driver interactions directly. This creates a catastrophic blind spot: prompt injection attacks that compromise the agent runtime can trivially evade all SDK-based logging, leaving no audit trail.

\textbf{5. Specs vs. Platforms: The Integration Tax.} OpenTelemetry GenAI and OpenInference provide data models but no storage or visualization—requiring additional backend integration. Conversely, SaaS platforms (Datadog, LangSmith) bundle the full stack but impose vendor lock-in through proprietary schemas and APIs.

\subsection{The Motivation for System-Level Observability}

Because current solutions live \emph{inside} the agent process, they inherit its fragility:

\begin{itemize}
  \item \textbf{Breakage on prompt evolution} – Each workflow change requires redeploying instrumentation.
  \item \textbf{Evasion by compromised agents} – Prompt injection can silence or falsify logs.
  \item \textbf{Blind to cross-process interactions} – Cannot see shell script execution, subprocess coordination, or external tool invocation.
\end{itemize}

System-level tracing at kernel and network boundaries addresses these gaps:

\begin{table}[h]
\centering
\small
\begin{tabular}{p{0.45\columnwidth} p{0.45\columnwidth}}
\toprule
\textbf{Where today's SDKs stop} & \textbf{What boundary tracing would still see} \\
\midrule
Missing span when agent spawns \texttt{curl} directly & \texttt{execve("curl", ...)} + network write \\
Agent mutates its own prompt string before logging & Raw ciphertext leaving the TLS socket \\
Sub-process mis-uses GPU & \texttt{ioctl} + CUDA driver calls \\
\bottomrule
\end{tabular}
\end{table}

In other words, existing tools solve the "what happened inside my code?" story; system-level tracing can answer "what actually hit the wire and the OS?"—a complementary, harder-to-tamper vantage point.

This gap represents a critical research opportunity for the community.

\section{Vision: A Two-Plane Architecture for Scalable Agent Observability}

Given the deployment challenges (security, cost, fragmentation) and the inadequacy of current solutions, we propose a fundamental architectural shift. Scalable agent observability requires two integrated layers:

\subsection{The Data Plane: Zero-Instrumentation Foundation}

\textbf{Core Principle:} Observability must be \emph{decoupled from agent internals}. Rather than instrumenting rapidly-evolving application code, capture telemetry at \emph{stable system boundaries}—kernel syscalls, network interfaces, and TLS endpoints.

\textbf{The Agent Stack Architecture.} Understanding where to observe requires understanding the agent architecture as three nested layers:


\begin{comment}
    

\begin{center}
    
\begin{Verbatim}[fontsize=\small, commandchars=\\\{\}]
┌───────────────────────────────────────────────┐
│          ☁  Rest of workspace / system       │
│  (APIs, DBs, message bus, OS, Kubernetes…)    │
│                                               │
│   ┌───────────────────────────────────────┐   │
│   │       Agent runtime / framework       │   │
│   │ (LangChain, claude-code, gemini-cli …)│   │
│   │  • orchestrates prompts & tool calls  │   │
│   │  • owns scratch memory / vector DB    │   │
│   └───────────────────────────────────────┘   │
│               ↑ outbound API calls            │
│───────────────────────────────────────────────│
│               ↓ inbound events                │
│   ┌───────────────────────────────────────┐   │
│   │          LLM serving provider         │   │
│   │    (OpenAI endpoint, local llama.cpp) │   │
│   └───────────────────────────────────────┘   │
└───────────────────────────────────────────────┘
\end{Verbatim}
\end{center}
\end{comment}


\begin{itemize}
  \item \textbf{LLM serving provider} – token generation, non-deterministic reasoning, chain-of-thought text that may or may not be surfaced. Most performance optimization work focuses on this layer.
  \item \textbf{Agent runtime layer} – orchestrates tasks into sequences of LLM calls plus external tool invocations; manages transient ``memories'' and state.
  \item \textbf{Outside world} – OS, containers, APIs, databases, other services and agents.
\end{itemize}

\textbf{The Observability Boundaries.} For observability purposes, the critical interfaces are:
\begin{itemize}
  \item \textbf{Network boundary}: TLS-encrypted JSON inference requests/responses between agent and LLM provider
  \item \textbf{System boundary}: Syscalls and subprocess creation when agents execute tools (\texttt{curl}, \texttt{grep}, file I/O)
  \item \textbf{Inter-agent boundary}: Message passing, shared memory, or API calls between coordinating agents
\end{itemize}

Traditional software observability is \textbf{instrumentation-first}—manually inserting logs, spans, and metrics into application code. But AI agents \emph{dynamically modify their own logic} through prompt evolution, spontaneous tool creation, and self-modification. This constant internal mutability makes instrumentation fundamentally fragile.

\textbf{Benefits of System-Boundary Observability:}
\begin{itemize}
  \item \textbf{Framework neutrality}: Works across all agent runtimes (LangChain, AutoGen, gemini-cli, custom frameworks).
  \item \textbf{Semantic stability}: Captures prompt-level semantics without chasing rapidly-changing framework APIs.
  \item \textbf{Trust \& auditability}: Independent audit trail that cannot be compromised by in-agent malware or prompt injection.
  \item \textbf{Universal causal graph}: Merges agent-level semantics (prompts, reasoning) with OS-level events (syscalls, network) into one coherent timeline.
  \item \textbf{Multi-vendor unification}: Provides single observability layer across fragmented stack (LLM SaaS, agent code, system runtime, tools).
\end{itemize}

\textbf{Technical Implementation.} Technologies like eBPF~\cite{brendangregg,ebpfio} enable this vision by providing kernel-level tracing with <3\% overhead~\cite{zheng2025extending}. Key capabilities include:
\begin{itemize}
  \item \textbf{TLS interception}: Hooking userspace TLS read/write functions to capture encrypted LLM traffic without proxies
  \item \textbf{Syscall tracing}: Monitoring \texttt{execve()}, file I/O, network calls to track tool execution
  \item \textbf{Process genealogy}: Tracking subprocess creation chains to understand agent-spawned workflows
\end{itemize}

\subsection{The Cognitive Plane: Agents Observing Agents}

\textbf{The Human Scalability Crisis.} Even with perfect telemetry from the Data Plane, human operators cannot effectively monitor thousands of production agents generating millions of events per hour. The sheer volume, combined with the semantic complexity (understanding reasoning chains, detecting subtle persona drift, correlating multi-agent failures), exceeds human cognitive capacity.

\textbf{Core Thesis:} The only way to observe complex, autonomous AI systems at scale is with \emph{other autonomous AI systems}. We propose a Cognitive Plane—an autonomous multi-agent system dedicated to observability itself.

\textbf{Architecture:} The Cognitive Plane consists of specialized observability agents~\cite{Rombaut2025Watson,Kim2025AgenticInterp}:

\begin{enumerate}
  \item \textbf{Observer Agents}: Continuously monitor telemetry streams from the Data Plane, applying learned patterns to detect anomalies (reasoning loops, cost spikes, security violations)~\cite{Dong2024AgentOps}.

  \item \textbf{Diagnoser Agents}: Perform automated root-cause analysis by correlating semantic traces (prompt evolution) with system events (syscalls, network failures). Use LLM reasoning to generate human-readable incident reports explaining \emph{why} a failure occurred~\cite{Moshkovich2025Pipeline}.

  \item \textbf{Remediator Agents}: Take autonomous corrective actions within defined safety boundaries (e.g., circuit-breaking runaway agents, adjusting prompt temperature, scaling resources). Escalate high-risk decisions to human operators via structured recommendations.

  \item \textbf{Cost Optimizer Agents}: Analyze token usage, API call patterns, and reasoning efficiency across all agents to identify optimization opportunities and automatically implement approved changes.
\end{enumerate}

\textbf{Key Properties:}
\begin{itemize}
  \item \textbf{Semantic understanding}: LLM-powered agents can interpret natural language prompts, reasoning chains, and tool interactions—bridging the semantic gap that rule-based systems cannot cross.

  \item \textbf{Continuous learning}: Observability agents learn from historical incidents, improving detection accuracy and expanding their understanding of normal vs. anomalous behavior.

  \item \textbf{Proactive management}: Rather than reactive debugging, the Cognitive Plane \emph{predicts} failures (e.g., detecting early signs of reasoning loops) and intervenes before user impact.
\end{itemize}

\section{Future Directions and Open Challenges}

\textbf{1. Causal Observability.} Current systems correlate events; future work must establish \emph{causal relationships}. Why did an agent make a specific decision? What would have happened with a different prompt? This requires causal inference techniques applied to agent telemetry.

\textbf{2. Evaluation Benchmarks.} The field lacks standardized benchmarks for agent observability systems. We need datasets of real agent failures, metrics for semantic coverage, and evaluation frameworks for observability agent effectiveness.

\textbf{3. Security of the Observability Layer.} If the Cognitive Plane is compromised, attackers gain visibility into all production agents. Research is needed on tamper-proof telemetry channels, secure multi-party computation for sensitive traces, and Byzantine-fault-tolerant observability architectures.

\textbf{4. The Semantic Gap.} While system-level tracing captures raw events, extracting agent \emph{intent} requires sophisticated NLP and program synthesis. How do we infer "the agent is trying to accomplish X" from syscall sequences and TLS payloads?

\textbf{5. Standardization.} OpenTelemetry semantic conventions for agents are nascent. The community needs consensus on span attributes (\texttt{agent.reasoning.loop\_id}, \texttt{tool.coordination.pattern}), trace topology (how to represent multi-agent interactions), and evaluation schemas.

\section{Conclusion}

The current fragmented, model-centric approach to AI agent observability represents a fundamental barrier to large-scale deployment. Application-level instrumentation creates brittle SDK silos that cannot address the security, cost, and complexity challenges inherent to agent systems. Traditional APM tools, LLM-centric monitoring, and serving observability each address only narrow slices of the problem, leaving critical blind spots.

We argue that the future of reliable, large-scale agentic AI depends on adopting a unified, two-plane observability architecture: (1) a \textbf{Data Plane} that provides zero-instrumentation telemetry capture at stable system boundaries, decoupling observability from rapidly-evolving agent internals while providing tamper-resistant audit trails, and (2) a \textbf{Cognitive Plane} where autonomous observability agents provide the semantic understanding and proactive management that human operators cannot achieve at scale.

This architectural vision directly addresses the deployment challenges we identified: system-boundary observability eliminates SDK fragmentation and security vulnerabilities; unified telemetry enables cross-layer cost optimization; and autonomous observability agents bridge the semantic gap between raw events and agent intent. Most importantly, this approach scales: as agent deployments grow from hundreds to millions, the observability system grows with them—not through human labor, but through autonomous agent-to-agent coordination.

The path forward requires sustained research effort across the community: developing causal inference techniques for agent behavior, establishing evaluation benchmarks, securing the observability infrastructure itself, and standardizing semantic conventions. But the fundamental insight is clear: \textbf{only agents can truly observe agents at scale}. The question is not whether this transition will occur, but how quickly we can build the foundations to make it possible.

\bibliographystyle{ACM-Reference-Format}
\bibliography{../ai}

\end{document}

